\documentclass[uplatex,dvipdfmx]{ujarticle}
\usepackage[dvipdfmx]{graphicx}
\usepackage{listings}
\usepackage{nameref}
\usepackage{plistings}
\usepackage{latexsym}
\usepackage{bm}
\usepackage{amsmath, amssymb}
\usepackage{here}
\usepackage{url}
\usepackage{color}

\newtheorem{Proof}{証明}
\def\qed{\hfill $\Box$}
\newcommand{\pr}[1]{\left(#1\right)}
\newcommand{\bk}[1]{\left[#1\right]}
\newcommand{\be}[1]{\left\{#1\right\}}
\newcommand{\ag}[1]{\langle#1\rangle}
\newcommand{\ve}[2]{\pr{\begin{array}{c}#1\\#2\end{array}}}
\newcommand{\ma}[2]{\pr{\begin{array}{c}#1\\\vdots\\#2\end{array}}}
\newcommand{\srcref}[1]{{\tt \nameref{#1}} (Listing \ref{#1}, p.\pageref{#1})}

\lstset{
    language=C,
	showstringspaces=false,
    basicstyle=\ttfamily\scriptsize,
    commentstyle=\textit\scriptsize\itshape\color[rgb]{0.5,0.7,0.5},
	keywordstyle=\bfseries\scriptsize\color[rgb]{0.8,0.2,0.4},
	%ndkeywordstyle=\bfseries\scriptsize\color[rgb]{0.2,0.2,0.8},
	stringstyle=\ttfamily\scriptsize\color[rgb]{0.9,0.4,0.2},
    frame=trbL,
    numbers=left,
    breaklines=true,
    title=\lstname,
	tabsize=4,
}

\lstset{
    language=bash,
	showstringspaces=false,
    basicstyle=\ttfamily\scriptsize,
    commentstyle=\textit\scriptsize\itshape\color[rgb]{0.5,0.7,0.5},
	keywordstyle=\bfseries\scriptsize\color[rgb]{0.8,0.2,0.4},
	%ndkeywordstyle=\bfseries\scriptsize\color[rgb]{0.2,0.2,0.8},
	stringstyle=\ttfamily\scriptsize\color[rgb]{0.9,0.4,0.2},
    frame=trbL,
    numbers=left,
    breaklines=true,
    title=\lstname,
	tabsize=4,
}

\begin{document}

\title{ESP32+FluidNC+Raspberry Pi 4B+CNCjs セットアップ草案}
\author{草案者: Wood Burned}
\date{\today}
\maketitle

\section{ESP32ボードの準備とファーム書き込み}

CNC用モーションコントローラとしてESP32を用いる。ファームウェアはFluidNCを採用する。初期導入はPC上のGoogle ChromeからFluidNC公式のオンラインインストーラを用いる。Web Serial APIを使うためChrome前提とする。

手順の概観は次の通り。
\begin{enumerate}
\item ESP32をUSBでPCに接続する(給電はPC側から行う)。
\item Google ChromeでFluidNC Web Installerを開き、Connectで該当シリアルポートを選択する。
\item InstallからESP32向けビルド(Wi-Fi有効版)を選択して書き込む。\footnote{運用ではWi-Fiは使わない。導入や更新時のみ使う方針とする。}
\item 書き込み完了後、\textbf{インストーラの「Config」(または「Files」)欄}から \texttt{config.yaml} を新規作成する。基本パラメータはここで対話的に調整する(手入力でYAMLを配布せず、インストーラ側で編集・保存する)。
\end{enumerate}

注意点として、各軸のSTEP/DIR/ENのピン割付は、ESP32で\textbf{出力に使ってよいGPIO}を選ぶ。ブートストラップ用のストラッピングピン(GPIO0, 2, 5, 12, 15)やフラッシュ接続ピン(GPIO6〜11)、入力専用ピン(GPIO34〜39)は避ける。UARTのTX/RX(GPIO1/3)も原則避ける。割付は後述の設定方針と合わせて決定する。

\begin{lstlisting}[caption=Chromeでの書き込みとconfig作成の流れ(要点), label=code:installer, language=bash]
1) Chromeでインストーラを開く -> Connect -> ポート選択
2) Install -> "ESP32 / Wi-Fi" を選んで書き込み
3) 再起動後、インストーラの "Config" 欄で config.yaml を新規作成・保存
- 軸やピン、移動量・速度・加速度などはこの画面から編集
4) 以降の編集・保存も同画面から行う(実機確認->再調整)

\end{lstlisting}

\section{FluidNCの基本設定(Config欄で行う方針)}

本書では、\textbf{基本パラメータを配布せず}、FluidNC Web Installerの\textbf{Config欄から対話的に調整}する。理由は、機械固有の要素(スクリューピッチ、マイクロステップ、剛性、負荷、ドライバ仕様)が初期値を大きく左右するためである。次の考え方で進める。

\subsection*{設定の流れ(対話的チューニング)}
\begin{enumerate}
\item 軸とモータの論理構成を定義する(X/Y/Z、スレーブ軸の有無など)。
\item 各軸の\textbf{ピン割付}を決める。前節の\textbf{使用可能GPIOの制約}を満たす組合せを選ぶ。
\item \textbf{steps\_per\_mm}は実測で校正する。10,mm, 50,mmの移動指令とダイヤルゲージ等での測長を繰り返し、誤差を詰める。
\item \textbf{max\_rate}と\textbf{acceleration}は安全側から立ち上げ、失歩・共振・発熱を観察しつつ段階的に上げる。
\item 方向反転やリミット/プローブ極性は、実機の挙動に合わせて修正する。
\end{enumerate}

\noindent 補足: Config欄はプレーンテキストでの保存を前提とする。エディタでのリッチテキスト貼付は避ける。

\section{Raspberry Pi 4BへのCNCjs導入(デスクトップ版OS + 公式Dockerfile)}

CNCjsはWebベースのG-code送受信/UIであり、Raspberry Pi 4Bで常用する。OSは\textbf{Raspberry Pi OS(64-bit, Desktop版)}を用いる。導入時のみWi-Fiを有効化し、\textbf{運用は有線(USB/有線LAN)}とする。「電プチ」(突然の電源断)対策として Overlay File System を有効化する運用を推奨する。

\subsection*{OSインストール(Raspberry Pi Imager)}
\begin{enumerate}
\item PCでRaspberry Pi Imagerを起動し、OSは\textbf{Raspberry Pi OS (64-bit) with desktop}を選ぶ。
\item Advanced Settings(歯車)で、ホスト名/ロケール、\textbf{SSH有効化}、\textbf{Wi-Fi設定(導入時のみ使用)}を事前指定して書き込む。
\item 初回起動後、必要なパッケージを更新する。
\end{enumerate}

\begin{lstlisting}[caption=初期パッケージ導入(APT), label=code:apt-init, language=bash]
sudo apt update
sudo apt -y full-upgrade
導入に使うツール群

sudo apt -y install git curl build-essential docker.io
導入後の運用は有線、Wi-Fiは無効化(必要に応じて)

sudo rfkill block wifi
\end{lstlisting}

\subsection*{電プチ対策(Overlay File System)}
Raspberry Pi OSは\texttt{raspi-config}から\textbf{Overlay File System}を有効化できる。運用中はrootfsを実質Read-Only化し、変更はRAM上に乗る。設定変更や更新時は一時的に無効化して再起動する運用にする。

\begin{lstlisting}[caption=OverlayFSの有効化(対話/非対話の例), label=code:overlay, language=bash]
対話(raspi-config)

sudo raspi-config # Performance Options -> Overlay File System -> Enable
非対話(新しめのraspi-configで有効)

sudo raspi-config nonint enable_overlayfs
sudo reboot
\end{lstlisting}

\subsection*{CNCjsの導入(公式Dockerfileを用いる)}

公式リポジトリのDockerfileを用いて\textbf{Dockerイメージをローカルビルド}し、コンテナとして常駐運用する。公式Dockerfileは\texttt{dist/cncjs}の成果物を前提とするため、\textbf{先にNode/Yarnでビルドする}。

\paragraph{1) CNCjsのビルド(dist生成)}
\begin{lstlisting}[caption=Node/Yarnでフロントをビルド(Pi上で可), label=code:build-dist, language=bash]
Node 18系を前提(bookworm既定のnodejsで可。必要ならNodeSource等を使用)

sudo apt -y install nodejs npm
sudo npm -g install yarn

git clone https://github.com/cncjs/cncjs.git
cd cncjs
yarn install
yarn run build # dist/cncjs が生成される
\end{lstlisting}

\paragraph{2) 公式Dockerfileでイメージ化}
\begin{lstlisting}[caption=Dockerビルドと起動, label=code:docker-run, language=bash]
リポジトリ直下にある公式Dockerfileを使用

docker build -t cncjs:local -f Dockerfile .
デフォルトExposeは 8000。ホスト8080に割り当てる例。
/dev/ttyUSB* (or /dev/ttyACM*) をコンテナへ渡す

docker run -d --name cncjs
--restart=unless-stopped
--device /dev/ttyUSB0
-p 8080:8000 cncjs:local
\end{lstlisting}

\paragraph{3) 起動・アクセス・運用方針}
\begin{itemize}
\item ブラウザから \texttt{http://<raspi-ip>:8080/} にアクセスし、CNCjs UIを操作する。ブラウザはGoogle Chromeを推奨する。
\item ESP32はUSB直結し、CNCjsのポート選択で \texttt{/dev/ttyUSB*} または \texttt{/dev/ttyACM*} を選ぶ。
\item 運用は有線前提、Wi-Fiは基本無効化(導入・更新時のみ有効)。
\end{itemize}

\begin{lstlisting}[caption=systemdで自動起動(コンテナを常駐させる例), label=code:systemd, language=bash]
/etc/systemd/system/cncjs.service

[Unit]
Description=CNCjs container
After=network-online.target docker.service
Wants=docker.service

[Service]
Restart=always
ExecStart=/usr/bin/docker start -a cncjs
ExecStop=/usr/bin/docker stop cncjs

[Install]
WantedBy=multi-user.target
有効化

sudo systemctl daemon-reload
sudo systemctl enable --now cncjs
\end{lstlisting}

\section{この草案の要点(実施順)}
\begin{enumerate}
\item Google ChromeでFluidNC Web Installerを使い、ESP32へ書き込み後、\textbf{インストーラのConfig欄}から \texttt{config.yaml} を作って編集する(配布YAMLは用いない)。
\item 各軸のピン割付はESP32のGPIO制約を満たす組合せにする(ストラッピング・フラッシュ・入力専用ピンは避ける)。
\item Raspberry Pi OSは\textbf{デスクトップ版}をRaspberry Pi Imagerで導入。導入時のみWi-Fi有効、以後は無効化。電プチ対策でOverlayFSを有効化。
\item CNCjsは\textbf{公式Dockerfile}でローカルビルドし、コンテナ運用する(デフォルトExposeは8000、例では8080へ割当)。
\end{enumerate}

\begin{thebibliography}{999}
\bibitem{fluidnc-installer} FluidNC Web Installer(公式).
\bibitem{fluidnc-config} FluidNC Config file Overview(公式Wiki).
\bibitem{fluidnc-motion} FluidNC Motion Setup(公式Wiki).
\bibitem{espressif-gpio} Espressif ESP-IDF: GPIO \& RTC GPIO(公式).
\bibitem{espressif-datasheet} Espressif ESP32 Series Datasheet(公式).
\bibitem{rpi-imager-wifi} Raspberry Pi Documentation: Wi-Fi/Imager Advanced Settings(公式).
\bibitem{rpi-overlay-wp} Raspberry Pi Whitepaper: Making a more resilient file system(公式).
\bibitem{cncjs-install} CNCjs Installation / Raspberry Pi Setup Guide(公式).
\bibitem{cncjs-dockerfile} CNCjs 公式Dockerfile(master).
\bibitem{dockerhub} Docker Hub: cncjs/cncjs(公式イメージ).
\end{thebibliography}

\end{document}