\documentclass[uplatex,dvipdfmx]{ujarticle}
\usepackage[dvipdfmx]{graphicx}
\usepackage{listings}
\usepackage{nameref}
\usepackage{plistings}
\usepackage{latexsym}
\usepackage{bm}
\usepackage{amsmath, amssymb}
\usepackage{here}
\usepackage{url}
\usepackage{color}

\newtheorem{Proof}{証明}
\def\qed{\hfill $\Box$}
\newcommand{\pr}[1]{\left(#1\right)}
\newcommand{\bk}[1]{\left[#1\right]}
\newcommand{\be}[1]{\left\{#1\right\}}
\newcommand{\ag}[1]{\langle#1\rangle}
\newcommand{\ve}[2]{\pr{\begin{array}{c}#1\\#2\end{array}}}
\newcommand{\ma}[2]{\pr{\begin{array}{c}#1\\\vdots\\#2\end{array}}}
\newcommand{\srcref}[1]{{\tt \nameref{#1}} (Listing \ref{#1}, p.\pageref{#1})}

\lstset{
    language=C,
	showstringspaces=false,
    basicstyle=\ttfamily\scriptsize,
    commentstyle=\textit\scriptsize\itshape\color[rgb]{0.5,0.7,0.5},
	keywordstyle=\bfseries\scriptsize\color[rgb]{0.8,0.2,0.4},
	%ndkeywordstyle=\bfseries\scriptsize\color[rgb]{0.2,0.2,0.8},
	stringstyle=\ttfamily\scriptsize\color[rgb]{0.9,0.4,0.2},
    frame=trbL,
    numbers=left,
    breaklines=true,
    title=\lstname,
	tabsize=4,
}

\lstset{
    language=bash,
	showstringspaces=false,
    basicstyle=\ttfamily\scriptsize,
    commentstyle=\textit\scriptsize\itshape\color[rgb]{0.5,0.7,0.5},
	keywordstyle=\bfseries\scriptsize\color[rgb]{0.8,0.2,0.4},
	%ndkeywordstyle=\bfseries\scriptsize\color[rgb]{0.2,0.2,0.8},
	stringstyle=\ttfamily\scriptsize\color[rgb]{0.9,0.4,0.2},
    frame=trbL,
    numbers=left,
    breaklines=true,
    title=\lstname,
	tabsize=4,
}

\begin{document}

\title{カーネルハック 課題1-3 中間レポート}
\author{学籍番号202211587 小木勇輝}
\date{\today}
\maketitle

\section{test}

loren ipsum

\begin{lstlisting}[caption=bash, label=code:fuga, language=bash]
$ cat /etc/os-release
PRETTY_NAME="Ubuntu 22.04.4 LTS"
NAME="Ubuntu"
VERSION_ID="22.04"
VERSION="22.04.4 LTS (Jammy Jellyfish)"
VERSION_CODENAME=jammy
ID=ubuntu
ID_LIKE=debian
HOME_URL="https://www.ubuntu.com/"
SUPPORT_URL="https://help.ubuntu.com/"
BUG_REPORT_URL="https://bugs.launchpad.net/ubuntu/"
PRIVACY_POLICY_URL="https://www.ubuntu.com/legal/terms-and-policies/privacy-policy"
UBUNTU_CODENAME=jammy

  \end{lstlisting}
   \begin{thebibliography}{999}
    \bibitem{machine}`mouse X5-R7 仕様詳細`、https://www.mouse-jp.co.jp/contents/other/old\_products/file/2021/note/mouse/2111\_mouse\_X5-R7\_2012X5-aR7RNIAR.pdf
    \bibitem{wiki}`UEFI SSD/カーネルのコンパイル`、https://www.coins.tsukuba.ac.jp/~yas/coins/slab-kernel-2024/hiki/?UEFI SSD/ホストOSでのカーネルのコンパイル
    \bibitem{qiita1} `組み込みLinuxデバイスドライバの作り方 (6)`、https://qiita.com/iwatake2222/items/ade0a73d4c05fc7961d3
    \bibitem{hanashin} `カーネルモジュールの作り方`、https://hana-shin.hatenablog.com/entry/2022/02/16/221814
    \bibitem{haikuhp} `Compiling for x86\_64`、https://www.haiku-os.org/guides/building/compiling-x86\_64
    \bibitem{haikuerror}`Error: you are using a Haiku clone without tags`、https://discuss.haiku-os.org/t/error-you-are-using-a-haiku-clone-without-tags/3920
    \bibitem{haikuSyscallPhony}`System calls`、https://www.haiku-os.org/documents/dev/system\_calls/
    \end{thebibliography}
  \end{document}
\end{document}
